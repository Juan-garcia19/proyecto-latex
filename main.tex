\documentclass{article}
\usepackage[utf8]{inputenc}
\usepackage[spanish]{babel}
\usepackage{listings}
\usepackage{graphicx}
\graphicspath{ {images/} }
\usepackage{cite}

\begin{document}

\begin{titlepage}
    \begin{center}
        \vspace*{1cm}
            
        \Huge
        \textbf{Trabajo calistenia}
            
        \vspace{0.5cm}
        \LARGE
        Parcial 1
            
        \vspace{1.5cm}
            
        \textbf{Juan Esteban García Durango}
            
        \vfill
            
        \vspace{0.8cm}
            
        \Large
        Despartamento de Ingeniería Electrónica y Telecomunicaciones\\
        Universidad de Antioquia\\
        Medellín\\
        Marzo de 2021
            
    \end{center}
\end{titlepage}

\tableofcontents
\newpage
\section{Estado inicial}\label{intro}
El estado inicial del ejercicio son las dos tarjetas unidas la una a la otra debajo de una hoja de papel en una superficie plana horizontal, es decir paralela al piso.

\section{Pasos a seguir} \label{contenido}
1) Coger suavemente la hoja sin arrugarla y ponerla en una superficie plana horizontal.\\
2)Agarrar las tarjetas con una mano sin despegarlas una de la otra.\\
3)Ubicar de manera vertical las dos tarjetas sin despegarlas la única de la otra en el centro de la hoja\\
4)Poner en el centro de la hoja el lado más corto de las tarjetas\\
5)con ayuda del dedo índice, sostener la unión de las dos tarjetas de la parte superior\\
6) Con ayuda del dedo pulgar y meñique coger la parte inferior de una de las tarjetas\\
7) Ir despegando poco a poco la parte inferior de las tarjetas sin separar la parte superior ubicadas por el dedo índice hasta formar un triángulo que se sostenga por si solo\\
8) En caso de derrumbarse el triángulo, volver al paso 3)\\


\end{document}
